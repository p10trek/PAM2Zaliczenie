\documentclass[12pt,a4paper]{article}
\usepackage[OT4]{polski}
\usepackage[utf8]{inputenc}
\usepackage{amssymb}
\usepackage[polish]{babel}
\usepackage{amsmath}
\usepackage{amsfonts}
\usepackage[left=3.5cm,right=2cm,top=2.5cm,bottom=2.5cm]{geometry}
\usepackage{graphicx}
\usepackage{indentfirst} 
\author{Karpiński Maciej\\Radlak Piotr\\Wiecheć Sebastian\\\\\\\\\includegraphics[width=0.7\linewidth]{img/logoPWSZ.jpg}\\\\\\\\Projektowanie i programowanie systemów internetowych I}
\title{Projekt systemu zleceń mafijnych\\Mafia 2.0}

\begin{document}

	%Stron tytułowa
	\pagenumbering{gobble}
	\maketitle
	\newpage

	%Spis treści
	\tableofcontents
	\pagenumbering{arabic}
	\newpage

	%Początek pierwszej sekcji opisującej system
	\section{Opis funkcjonalny systemu}
		Opis opis opis
	
	%Sekcja druga, opis wdrożonych kwalifikacji
	\section{Wdrożone kwalifikacje}
		Wdrożone kwalifikacje z punkt 1.1 zasad projektu z opisem
	
	%sekcja trzecia, opis wykorzystanych technologii
	\section{Opis technologiczny}
		Przy tworzeniu projektu Mafia 2.0 wykorzystano technologie:

		\subsection{C\#}
			C\# jest obiektowym językiem programowania, zaprojektowanym w latach 1998 – 2001 dla firmy Microsoft.
			Napisany program jest kompilowany do Common Intermediate Language(CLI), który następnie wykonywany jest w środowisku uruchomieniowym takim jak .NET Framework,
			.NET Core, Mono lub DotGNU.
			Wykorzystanie CLI sprawia, że kod programu jest wieleplatformowy (dopóki istnieje odpowiednie środowisko uruchomieniowe).
			C\# posiada wiele wspólnych cech z językami Object Pascal, Delphi, C++ i Java a najważniejszymi cechami C\# są:
			\begin{itemize}
				\item Obiektowość z hierarchią o jednym elemencie nadrzędnym (podobnie jak w Javie);
				\item Zarządzaniem pamięcią zajmuje się środowisko uruchomieniowe;
				\item Właściwości i indeksery;
				\item Delegaty i zdarzenia – rozwinięcie wskaźników C++;
				\item Typy ogólne, generyczne, częściowe, Nullable, domniemane, anonimowe;
				\item Dynamiczne tworzenie kodu;
				\item Metody anonimowe;
				\item Wyrażenia lambda.
			\end{itemize}
			
		\subsection{ASP.NET Core}
			ASP.Net Core jest wysokowydajnym frameworkiem, do budowania nowoczesnych aplikacji internetowych wykorzystujących moc obliczeniową chmur. ASP.Net Core jest technologią open - source,
			wykorzystującą silnik html Razor, dzięki której możliwe jest tworzenie aplikacji mulitplaformowych, które mogą być używane na każdym urządzeniu wyposażonym w przeglądarkę
			internetową.
			
		\subsection{Bootstrap}		
			Bootstrap jest frameworkiem CSS, który koncentruje się na uproszczeniu tworzenia frontendu stron internetowych. Rezultatem dodania Bootstrapa do projektu jest jednolity wygląd
			wszystkich elementów interfejsów we wszystkich przeglądarkach. Dodatkowo programiści mogą skorzystać z klas w CSS w celu dalszego dostosowywania wyglądu ich zawartości. Bootstrap
			zawiera kilka składników JavaScript, które zapewniają dodatkowe elementy interfejsu użytkownika, takie jak okna dialogowe, podpowiedzi czy karuzele. 

		\subsection{Entity Framework}		 
		 Entity Framework jest technologią open source do mapowania obiektowo – relacyjnego (ORM), które wspierają rozwój aplikacji zorientowanych na dane.
		 Entity Framework umożliwia programistom pracę z danymi w postaci obiektów i właściwości specyficznych dla domeny, bez konieczności przejmowania się bazowymi
		 tabelami i kolumnami baz danych, w których dane są przechowywane. 

		\subsection{MSSQL}		 
		 	Microsoft SQL Server jest systemem zarządzania relacyjnymi bazami danych opracowany przez firmę Microsoft. Cechą charakterystyczną jest wykorzystanie głównie język zapytań
		 	Transact-SQL, który jest rozwinięciem standardu ANSI/ISO. W projekcie wykorzystano wersje 2019 Express, która jest bezpłatną edycją programu Microsoft SQL Server, oferująca
		 	podstawowy silnik bazy danych, nieposiadający ograniczenia ilości obsługiwanych baz danych lub użytkowników. Ograniczenia, występujące w wersji Express to  m. in.:
		 	korzystanie z jednego procesora, 1 GB pamięci RAM, 10GB plików bazy danych czy brak SQL Agent.
	
	%Sekcja czwarta, instrukcja lokalnego i zdalnego uruchomienia systemu
	\section{Instrukcja lokalnego i zdalnego uruchomienia systemu}
		\subsection{Lokalne uruchomienie systemu}
			Do uruchomienia Systemu zleceń mafijnych Mafia 2.0 wymagane jest następujące oprogramowanie:
			\begin{itemize}
				\item Microsoft SQL Server 2019 Express
				\item SQL Server Management Studio
			\end{itemize}
		\subsection{Zdalne uruchomienie systemu}
			Opis zdalnego uruchomienia systemu
	
	\section{Wnioski projektowe}
		Wnioski
\end{document}